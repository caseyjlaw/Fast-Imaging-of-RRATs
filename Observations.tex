\section{Observations}

We observed all three fields with 10 millisecond integrations, which was the fastest integration time available at the VLA at the time. In this observing mode, the VLA recorded two polarizations (RR, LL) and 128, 2-MHz wide spectral channels. The 128 channels were configured in two spectral windows of 64 channels each starting at 1372 and 1756 MHz. In this mode, we recorded data at a rate of 75 MB s$^{-1}$. The typical 10$\sigma$ limiting sensitivity of the VLA in this correlator configuration is 100 mJy, equivalent to a fluence limit of 1 Jy ms.

The three RRATs we observed (see Table \ref{tab:obs}) were chosen for a variety of reasons. First, we observed RRAT J0628+09 early in the program as a demonstration of the new correlator mode and transient search pipeline. This RRAT pulses relatively frequently \citep[141 bursts per hour seen by Arecibo;][]{2009ApJ...703.2259D}, but at the time had not yet been localized more precisely than a few arcminutes. RRATs J1925-16 and J1911+00 pulsed much less frequently, but had produced pulses bright enough to be seen by the fast imaging mode available at the time. We observed J1911+00 the longest of all three targets, because the field had also been observed by the \emph{Chandra X-ray Observatory}.

\begin{table} 
    \begin{tabular}{ c c c c }
        Date & RRAT & Antenna Configuration & Duration \\ 
         23 April 2012 & J0628+09 & C & 30 min \\ 
         26 December 2012 -- 1 January 2013 & J1925-16 & A & 4 hours \\ 
         14 December 2012 -- 1 January 2013 & J1911+00 & A & 12 hours \\
    \end{tabular} 
    \caption{Observations of RRATs \label{tab:obs}} 
\end{table}
 
  
  
  