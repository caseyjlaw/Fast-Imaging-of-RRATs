\section{Introduction}

Since the discovery of the first pulsing neutron star (NS), radio observations have steadily revealed a wide range of transient phenomena from NSs. Normal pulsars \ref{HEWISH_1969}, millisecond pulsars (Backer et al. 1982), and mag- netars (Camilo et al. 2006) all have distinct radio transient behavior. Understanding what this tran- sient phenomenology means physically has led to general models for NSs (e.g., Perna & Pons 2011).
A new manifestation of the NS as a transient is the Rotating Radio Transient (RRAT; McLaugh- lin et al. 2006). RRATs were discovered by search- ing large, single-dish radio surveys for individual millisecond-long pulses. Several surveys have now identified a few dozen RRATs, but their relation to the larger population of NSs is not clear. Since the radio emission represents a trivial part of the energy emitted by NSs, it is difficult to develop physical models from radio phenomenology. As described below, localizing the radio source and finding its X-ray counterpart will help understand how RRATs are physically like or unlike other NS transients.
To understand the nature of these unusual ob- jects, we propose a novel EVLA observation to precisely measure the locations of four RRATs. This proposal can be carried out as part of an ap- proved RSRO for one of us (CJL) to enable high data rate and fast imaging science. In §2, we de- scribe the nature of RRATs and the need to iden- tify their origin. In §3, we describe our efforts to develop a new concept for observing with radio interferometers that we call “fast imaging”. We show EVLA fast imaging can address open ques- tions about the nature RRATs.
  
  