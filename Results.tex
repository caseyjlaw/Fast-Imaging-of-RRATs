\section{Results}

\subsection{J1925-16}
\subsubsection{RRAT detection}
\label{rrat1925}

Figures \ref{fig:dmt} and \ref{fig:impeak} show the candidates detected in a single hour of observing toward RRAT J1925--16. Many candidates are seen at sub-second timescales and a few candidates are seen at the 10 ms integration time of this observing mode. All sub-second transients are localized in the image to source Jxx-xx...

At faster timescales, two clusters of candidates are localized to (RA, Dec) = (xx, xx). 

Counterparts...

The time between pulses... Minimum period...?

\subsubsection{Fast variable}
\label{fast}

**integrate down and push through casa?**

Location and association with PMNxxx...

Lightcurve...

Dynamic spectrum...

Secondary spectrum?

\subsection{J1911+00}

Despite the most observing time on RRAT J1911+00, no significant pulses were found...

Limit on pulse rate...
(If strong constraint, did RRAT turn off...?)

\subsection{J0628+09}

The detection and localization of J0628+09 was first reported in \cite{Law_2012}. Since the transient search software was redesigned for improved efficiency. A new, more computationally demanding search was made of the same data set and discovered more pulses from the J0628+09...

Observations were made in C configuration, with maximum baselines of roughly 3 km. We imaged these data with a beam size was 6.9\arcsec x 10.3\arcsec pixelated to cover xx times half-power width...

Number of pulses...

Minimum period analysis...

Image localization via stacking...

Background source LC for comparison with fast variable near RRAT J1925...
  
  