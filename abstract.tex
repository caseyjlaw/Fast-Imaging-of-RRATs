\begin{abstract}

We use the results of 10 millisecond imaging campaign with the Karl G. Jansky Very Large Array (VLA) to localize a set of transient neutron stars known as rotating radio transients (RRATs). RRATs pulse irregularly and rarely, which makes it difficult to localize them with time-domain techniques. We use the VLA to image dedispersed visibilities and localize the radio transients to arcsecond precision to enable the search for multi-wavelength counterparts. In 15 hours of observing, we detected pulses and localized two RRATs (J0628+09, J1925-16) and failed to detect pulses from another (J1911+00). We also report the serendipitous detection of scintillation of a bright background radio source, Jxxxx-xx.
  
\end{abstract}

  
  