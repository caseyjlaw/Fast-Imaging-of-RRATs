\section{Introduction}

Since the discovery of the first pulsing neutron star (NS), radio observations have steadily revealed a wide range of transient phenomena. Normal pulsars \citep{HEWISH_1969}, millisecond pulsars \citep{Backer_1982}, and magnetars \citep{Camilo_2006} all have distinct radio transient behavior. This transient phenomenology has been used to define general physical models for NSs \citep{Perna_2011}.

A new manifestation of the NS as a transient is the Rotating Radio Transient \citep[RRAT;][]{McLaughlin_2006}. RRATs were discovered by searching large, single-dish pulsar surveys for individual millisecond-long pulses. Several dozen RRATs are now known \citep{Deneva_2009, Burke_Spolaor_2010, Keane_2011} and all of them have dispersion measures consistent with being in our galaxy. RRATs sporadically repeat, but many have pulsed often enough that an underlying rotation period of the neutron star can be measured. Implied rotation periods are similar a few to several seconds, similar to slow pulsars and magnetars.

The rarity of their pulses argues that RRATs are at least as numerous as ordinary pulsars, which raises the "birthrate problem" \citep{Keane_2011}. Since the radio emission represents a trivial part of the energy emitted by NSs, it is difficult to develop physical models from radio phenomenology. \citep{Weltevrede_2006} note that the pulse energy distribution of realtively "normal" pulsars can make them appear to be RRATs, if it cannot be detected by Fourier time-domain techniques. However, one of the first RRATs localized was found to be associated with an rather unusual X-ray-luminous magnetar \citep{Lyne_2009}. Localizing the radio source and finding its X-ray counterpart will help understand the physical connection between RRATs and other NS transients.

To understand the nature of these unusual objects, we commissioned the VLA for a novel new observing mode known as "fast imaging" \citep{Law_2011}. Fast imaging requires generating millisecond-scale visibility data from correlators on radio interferometers.
This time scale is useful because it has sensitivity to fast transients and can measure dispersion at cm wavelengths. Using an interferometer expands on science done with single-dish telescopes via their precise (arcsecond) localization, high survey speed, and improved calibration and interference rejection. However, it requires managing data generated at roughly 1 TB hour$^{-1}$ and developing custom data analysis software.

We have applied this new observing mode in a 16-hour campaign to search for and localize three RRATs: J0628+09, J1925-16, and J1911+00. We ran a search pipeline on an NRAO compute cluster and detected pulses from the first J0628+09 and J1925-16. The first of these detections has been described elsewhere \citep{Law_2012}. Here, we describe the analysis of the entire 15 hour dataset.

  
  
  
  
  
  
  
  
  
  
  
  