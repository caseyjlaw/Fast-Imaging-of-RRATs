\section{Introduction}

Since the discovery of the first pulsing neutron star (NS), radio observations have steadily revealed a wide range of transient phenomena from NSs. Normal pulsars \ref{HEWISH_1969}, millisecond pulsars \ref{Backer_1982}, and magnetars \ref{Camilo_2006} all have distinct radio transient behavior. Understanding what this transient phenomenology means physically has led to general models for NSs \ref{Perna_2011}.

A new manifestation of the NS as a transient is the Rotating Radio Transient (RRAT; \ref{McLaughlin_2006}). RRATs were discovered by searching large, single-dish pulsar surveys for individual millisecond-long pulses; several dozen are now known (\ref{Deneva_2009}, \ref{Burke_Spolaor_2010}, \ref{Keane_2011}, \url{http://astro.phys.wvu.edu/rratalog}). Typical RRATs pulses have Galactic dispersion measures and sporadically repeat, tracing the underlying rotation period of the neutron star. Implied rotation periods are similar a few to several seconds, similar to slow pulsars and magnetars, but their physical nature is not yet known. 

The rarity of their pulses argues that RRATs are at least as numerous as ordinary pulsars, which raises the "birthrate problem" \ref{Keane_2011}. Since the radio emission represents a trivial part of the energy emitted by NSs, it is difficult to develop physical models from radio phenomenology. \ref{Weltevrede_2006} note that the pulse energy distribution of realtively "normal" pulsars can make them appear to be RRATs, if it cannot be detected by Fourier time-domain techniques. Localizing the radio source and finding its X-ray counterpart will help understand how RRATs are physically like or unlike other NS transients. Confusingly, one of the first RRATs localized was found to be associated with an rather unusual X-ray-luminous magnetar \ref{Lyne_2009}.

To understand the nature of these unusual objects, we commissioned the VLA for a novel new observing mode known as "fast imaging" (\ref{Law_2011}). Fast imaging is the use of correlators on radio interferometers for imaging on time scales faster than approximately 1 second. This time scale is useful because it has sensitivity to fast transients and can measure dispersion at cm wavelengths. Using an interferometer expands on science done with single-dish telescopes via their precise (arcsecond) localization, high survey speed, and improved calibration and interference rejection.

Using this new capability at the VLA, we conducted a 15-hour campaign to search for and localize three RRATs: J0628+09, J1925-16, and J1911+00. We ran a search pipeline on an NRAO compute cluster and detected pulses from the first two RRATs. The result of the first RRAT detected, J0628+09, has been described elsewhere (\ref{Law_2012}). Here, we describe the analysis of the entire 15 hour dataset, including a second RRAT localization and a fast scintillating radio source.

  
  
  
  
  